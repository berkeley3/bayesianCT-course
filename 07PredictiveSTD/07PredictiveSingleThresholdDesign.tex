\documentclass{beamer}

\mode<presentation> {
}

\title[]{Bayesian Clinical Trials} 
\subtitle{Single Threshold Design Extension} 
\date{} 

\usepackage{graphicx} 
\usepackage{booktabs} 
\usepackage{longtable} 
 \usepackage{hyperref}


\usepackage{color}
\usepackage{fancyvrb}

\definecolor{shadecolor}{gray}{0.95}

\DefineShortVerb[commandchars=\\\{\}]{\|}
\DefineVerbatimEnvironment{Highlighting}{Verbatim}{commandchars=\\\{\}}
\newenvironment{Shaded}{}{}
\newcommand{\KeywordTok}[1]{\textcolor[rgb]{0.00,0.44,0.13}{\textbf{{#1}}}}
\newcommand{\DataTypeTok}[1]{\textcolor[rgb]{0.56,0.13,0.00}{{#1}}}
\newcommand{\DecValTok}[1]{\textcolor[rgb]{0.25,0.63,0.44}{{#1}}}
\newcommand{\BaseNTok}[1]{\textcolor[rgb]{0.25,0.63,0.44}{{#1}}}
\newcommand{\FloatTok}[1]{\textcolor[rgb]{0.25,0.63,0.44}{{#1}}}
\newcommand{\CharTok}[1]{\textcolor[rgb]{0.25,0.44,0.63}{{#1}}}
\newcommand{\StringTok}[1]{\textcolor[rgb]{0.25,0.44,0.63}{{#1}}}
\newcommand{\CommentTok}[1]{\textcolor[rgb]{0.38,0.63,0.69}{\textit{{#1}}}}
\newcommand{\OtherTok}[1]{\textcolor[rgb]{0.00,0.44,0.13}{{#1}}}
\newcommand{\AlertTok}[1]{\textcolor[rgb]{1.00,0.00,0.00}{\textbf{{#1}}}}
\newcommand{\FunctionTok}[1]{\textcolor[rgb]{0.02,0.16,0.49}{{#1}}}
\newcommand{\RegionMarkerTok}[1]{{#1}}
\newcommand{\ErrorTok}[1]{\textcolor[rgb]{1.00,0.00,0.00}{\textbf{{#1}}}}
\newcommand{\NormalTok}[1]{{#1}}

\hypersetup{breaklinks=true, pdfborder={0 0 0}}
\setlength{\parindent}{0pt}
\setlength{\parskip}{6pt plus 2pt minus 1pt}
\setlength{\emergencystretch}{3em}  
\setcounter{secnumdepth}{0}
%\EndDefineVerbatimEnvironment{Highlighting}


\begin{document}



\begin{frame}
\titlepage % Print the title page as the first slide
\end{frame}


\begin{frame}{Single Threshold Design Extension}

\begin{itemize}[<+->]
\itemsep1pt\parskip0pt\parsep0pt
\item
  Single Threshold Design
\item
  Single Threshold Design can be extended using different kinds of
  informative prior distributions
\item
  Bayesian modification of the Simon's design to control frequentist
  error rates and comparison with Tan and Machin's STD
\item
  a Bayesian two-stage design based on the pre-experimental control of
  the probability of having a large posterior probability that the true
  response rate exceeds a target value.
\item
  this probability is computed with respect to the prior predictive
  distribution of the data: this design can be considered a predictive
  version of the STD
\end{itemize}

\end{frame}

\begin{frame}{Two-stage design}

\(n_1\) patients,
\(\quad Y=\left\{\begin{array}{ll} 1 & \rm{responder\ to\ drug}\\ 0 & \rm{otherwise}\end{array}\right.\)

\begin{itemize}
\itemsep1pt\parskip0pt\parsep0pt
\item
  \textbf{First stage} \[
  s_1=\sum_{i=1}^{n_1} y_i \left\{\begin{array}{ll}\leq r_1 & \rm{experiment\ stops}\\
  >r_1 & \rm{enrolled\ additional}\ n_2  \rm{\ patients \ for\ 2nd\ stage}\end{array}\right.
  \]
\end{itemize}

-\textbf{Second stage} \[
s=\sum_{i=1}^{n_1+n_2}y_i\left\{\begin{array}{ll}\leq r & \rm{experiment\ stops}\\
    >r & \rm{drug\ candidate\ for\ phase\ III}
\end{array}\right.
\]

\end{frame}



\begin{frame}{The predictive Single Threshold Design}

\begin{itemize}
\itemsep1pt\parskip0pt\parsep0pt
\item
  It takes the random nature of the data into account
\item
  the posterior probabilities that \(\theta\) exceeds \(\theta_u\) in
  the two stages are considered random since they are functions of S1
  and S, which are random variables
\end{itemize}

\end{frame}

\begin{frame}{Predictive two-stage design}

\begin{itemize}
\item
  Given the number of respondents \(s_1\) at the end of stage 1, the
  posterior distribution of \(\theta\) is \[
  \pi_{n_1}(\theta \vert S_1=s_1)\propto \pi(\theta)\times{\rm Binomial}(s_1; n_1, \theta)
  \] which is a \({\rm Beta}(\alpha+s_1, \beta+n_1-s_1)\) in the usual
  conjugate analysis.
\item
  Given the total number of respondents \(s\) at the end of stage 2, the
  posterior distribution of \(\theta\) is \[
  \pi_n(\theta \vert S_1>r_1, S=s)\propto \pi(\theta\vert S_1>r_1)\times{\rm Binomial}(s; n, \theta)
  \] which is still \({\rm Beta}(\alpha+s, \beta+n-s)\) in a conjugate
  analysis.
\end{itemize}

\end{frame}



\begin{frame}{The single threshold design (Tan and Machin 2002)}

-\textbf{Objective}: to choose the minimun sample size such that the
posterior probability \(\theta>\theta_u\) is greater or equal to a fixed
threshold when the response rate is equal to \(\theta_u\) + some small
value \(\epsilon>0\)

-Drug passes phase II if \(\theta> \theta_u\) (target value) if:

\[\left\{\begin{array}{ll}\min n_1 & \pi_{n_1}(\theta >\theta_u \vert  S_1=(\theta_u+\epsilon)n_1)\geq\lambda_1\\
 & \\
\min n & \pi_n(\theta>\theta_u\vert S=(\theta_u+\epsilon)n)\geq\lambda_2
\end{array}\right.
\]

where \(\lambda_1\) and \(\lambda_2\) are fixed probability thresholds.

\end{frame}



\begin{frame}{Bayesian sample size determination}

Bayesian sample size determination is a form of pre-posterior analysis,
i.e.~assessment of the value of data before they become available, in
which the prior distribution role is twofold:

\begin{itemize}
\item
  to obtain \(\pi(\theta\vert S)\) for posterior analysis
\item
  to define the marginal (predictive) distribution
  \(m(\theta; S)=\int_{0}^{1}{\rm Binomial}(s; n, \theta)\times \pi(\theta) d\theta\)
  for pre-posterior analysis
\item
  Following Wang and Gelfand 2002, the fitting or analysis prior
  \(\pi(\cdot\vert S)\) and the sampling or design prior \(m(\cdot;S)\) do
  not necesarily have to coincide.
\end{itemize}

\end{frame}


\begin{frame}{Predictive version of the STD (Sambucini, 2008)}

\begin{itemize}
\item
  The posterior probabilities that \(\theta>\theta_u\) in the two stages
  are considered random since they are function of \(S_1\) and \(S\),
  which are random variables
\item
  Determine two fixed thresholds probabilities \(\gamma_1\) and
  \(\gamma_2\) such that
\end{itemize}


\begin{align}
\mathbb{P}\left[\pi_{n_1}(\theta >\theta_u \vert S_1=(\theta_u+\epsilon)n_1)\geq\lambda_1\right]\geq\gamma_1
\end{align}


and

\begin{align}
\mathbb{P}\left[\pi_n(\theta>\theta_u\vert S=(\theta_u+\epsilon)n)\geq\lambda_2\right]\geq\gamma_2
\end{align}


where \(\mathbb{P}\) is the probability measure corresponding to the
predictive (or marginal) distribution of the data

\end{frame}




\begin{frame}{Analysis and Design prior}

\begin{itemize}
\itemsep1pt\parskip0pt\parsep0pt
\item
  \textbf{Analysis prior:} embodies the effective uncertainty on
  \(\theta\) (it is used to form the posterior distribution for making
  inference)
\item
  \textbf{Design prior:} describes a scenario under which a sensible
  sample size is established a priori according to a design criterion.
\end{itemize}

{ \textbf{Objective:}} use of a subjective prior based on elicitation of
expert opinions for design prior.

\end{frame}


\begin{frame}{The predictie STD design \(1^{st}\) stage}

Given the target response rate \(\theta_u\), consider the r.v.
\(\pi_{n_1}(\theta > \theta_u \vert S_1)\)

\begin{itemize}
\item
  given the probability thresholds \((\lambda_1,\gamma_1)\), select the
  smallest sample size \(n_1^*\) such that \(\forall n_1>n_1^*\) \[
  \mathbb{P}[\pi_{n_1}(\theta >\theta_u\vert S_1)\geq\lambda_1]\geq\gamma_1
  \]
\item
  \(\mathbb{P}\) is the probability measure corresponding to the prior
  predictive distribution of \(S_1\) induced by the design (sampling)
  prior: \[
  \mathbb{P}[\pi_{n_1}(\theta >\theta_u\vert S_1)\geq\lambda_1]=\sum_{s_1=\tilde{r}_1}^{n_1}m(S_1)
  \]
\end{itemize}

where \(\tilde{r}_1\) is the smallest \(s_1\) such that
\(\pi_{n_1}(\theta >\theta_u\vert s_1)\geq\lambda_1\).

\begin{itemize}
\itemsep1pt\parskip0pt\parsep0pt
\item
  Once the optimal value \(n^*_1\) is selected, then
  \(r^*_1=\tilde{r}_1-1\)
\end{itemize}

\end{frame}




\begin{frame}{The predictive STD design \(2^{nd}\) stage}

Given the target response rate \(\theta_u\), consider the r.v.
\(\pi_{n_1}(\theta > \theta_u \vert S)\)

\begin{itemize}
\item
  given the probability thresholds \((\lambda_2,\gamma_2)\), select the
  smallest \(n*\) such that \(\forall n>n^*\) \[
  \mathbb{P}[\pi_n(\theta >\theta_u\vert S)\geq\lambda_2]\geq\gamma_2
  \]
\item
  \(\mathbb{P}\) is the probability measure corresponding to the prior
  predictive distribution of \(S\) induced by the design (sampling)
  prior: \[
  \mathbb{P}[\pi_{n}(\theta >\theta_u\vert S)\geq\lambda_2]=\sum_{s=\tilde{r}}^{n}m(s\vert S_1>r^*)
  \]
\end{itemize}

where \(\tilde{r}\) is the smallest \(s\) such that
\(\pi_{n}(\theta >\theta_u\vert s)\geq\lambda_2\).

\begin{itemize}
\itemsep1pt\parskip0pt\parsep0pt
\item
  Once the optimal value \(n^*\) is selected, then \(r^*=\tilde{r}-1\)
\end{itemize}

\end{frame}


\begin{frame}{Examples}

-Using the R routines find sample size at both stages for
\(\theta_u=0.3\) when - analysis prior is non informative
(\(\pi_A=\theta_u-0.1\)) - design prior is an optimistic prior
(\(\pi_D=\theta_u+0.1\)) - for \(n_D=1\) and \(n_D=10\) - under
\(\lambda_1=0.6\), \(\gamma_1=0.6\) - under \(\lambda_2=0.8\),
\(\gamma_2=0.9\)

\end{frame}



\section{Case study}\label{case-study}

\begin{frame}{Illustrative example}

Let's consider the phase II clinical trial conducted by Foo et al.
{[}17{]} at the National Cancer Centre in Singapore to evaluate the
activity of gemcitabine in patients with metastatic nasopharyngeal
carcinoma and previously treated with chemotherapy.

In a previous study, using a Simon's minimax design with
\(\{\theta_0=0.05, \theta_u=0.2, \alpha=0.05, \beta=0.2\}\) find the
recommended two stage sample size

\end{frame}



\begin{frame}{Actual data}

\begin{itemize}
\item
  The actual data showed seven responders out of the 13 patients in the
  first stage
\item
  Therefore, the trial continued to the second stage, obtaining a
  cumulative number of 13 responders out of the total number (27
  patients)
\item
  Suppose now that we are interested in planning a new two-stage study
  to analyze the activity of gemcitabine. We can consider the data as a
  source of prior information
\end{itemize}

\end{frame}


\begin{frame}{Prior information on \(\theta_u\)}

\begin{itemize}
\item
  Since the results of this previous study show a strong efficacy of the
  gemcitabine, in the new study we could specify a target value
  \(\theta_u\) greater than 0.2 (the target response rate previously
  considered)
\item
  The elicitation of the design prior can be based on the available
  actual data (13 responders out of 27 patients at the end of the second
  stage):

  \begin{itemize}
  \itemsep1pt\parskip0pt\parsep0pt
  \item
    prior sample size \(n_D=27\) and a observed response rate
    \(pi_0^D=\frac{13}{27}\)
  \item
    using a non-informative beta distribution as analysis prior (e.g.
    \(\pi_0^{A}=\theta_u-0.1\)
  \end{itemize}
\end{itemize}

Under this assumptions find the recommended sample size when
\((\lambda_1,\gamma_1, \lambda_2,\gamma_2)=(0.7,0.7,0.8,0.8)\)

\end{frame}





\end{document}